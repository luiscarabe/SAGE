\documentclass{article}
% pre\'ambulo

\usepackage{lmodern}
\usepackage[T1]{fontenc}
\usepackage[spanish,activeacute]{babel}
\usepackage{mathtools}
\usepackage{amsmath}
\usepackage{amssymb}
\usepackage{amsthm}
\usepackage{siunitx}
\newcommand\tab[1][0.5cm]{\hspace*{#1}}
\title{Prepr\'actica 7}
\author{}

\begin{document}


\maketitle

Para el \textbf{Circuito 0} tenemos, por el principio de cortocircuito virtual, tenemos que $V_+ = V_- = V_{in}$ \newline \tab Adem'as, sabemos que $I_+ = I_- = 0 \si{A}$  \newline \tab Si llamamos \textit{I} a la corriente que circula por $ R_1$ y $R_2$ (de $R_1$ hacia $R_2$), tenemos:\

\begin{equation*}
\begin{gathered}
  V_o = V_{in} - IR_2 = V_{in}\cdot \left(1+ \frac{R_2}{R_1}\right)  \Longrightarrow \\
 A_v = \frac{V_o}{V_{in}} = 1+ \frac{R_2}{R_1}
\end{gathered}
\end{equation*}

\tab Se puede ver que la ganancia $A_V$ no depende de la frecuencia de la se'nal $V_{in}$, por lo cual su m'odulo tampoco. Adem'as, el desfase entre $V_{in}$ y $V_{out}$ es 0.\newline \newline
\tab El \textbf{Circuito 1 }es exactamente igual que el primero, pero tenemos en este caso un valor para $V_{in}$ que depende de $R_L, V_3$ y una impedancia capacitiva $Z_C$.\newline\tab
En este caso , si llamamos \textit{I} a la corriente que circula hacia $Z_C$ desde $R_L$ tenemos que 
\begin{equation*}
\begin{gathered}
V_{in} := V_{+} =  Z_C I = Z_C\frac{V_3}{R_L + Z_C} \\
A_v = \frac{V_{out}}{V_3} =  \frac{Z_C}{R_L + Z_C} \left(1+ \frac{R_2}{R_1}\right)
\end{gathered}
\end{equation*} 

\end {document}